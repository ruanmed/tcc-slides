\begin{table}[H]
    \centering
    \caption{Soluções encontradas de participantes da competição DB\_BOTGENDER.}
    \begin{tabular}{|c|l|l|}
        \hline
        \textbf{Posição}  
        & \makecell[l]{\textbf{Equipe}}
        & \makecell[l]{\textbf{Repositório de código no site \url{https://github.com/}}}
        \\ \hline
        7
        & Ipsas & Popescu 
        & \hyperlink{https://github.com/adiIspas/Bots-Gender-Profiling/}{/adiIspas/Bots-Gender-Profiling/}
        \\ \hline
        10
        & Goubin 
        & \hyperlink{https://github.com/pan-webis-de/goubin19/}{/pan-webis-de/goubin19/}
        \\ \hline
        11
        & Polignana & de Pinto 
        & \hyperlink{https://github.com/pan-webis-de/polignano19/}{/pan-webis-de/polignano19/}
        \\ \hline
        23
        & De La Peña & Prieto 
        & \hyperlink{https://github.com/JoseRPrietoF/autoria/}{/JoseRPrietoF/autoria/}
        \\ \hline
        30
        & Rahgoyu 
        & \hyperlink{https://github.com/HamedBabaei/PAN2019_bots_gender_profiling/}{/HamedBabaei/PAN2019_bots_gender_profiling/}
        \\ \hline
        32
        & Pryzybyla Przybyła 
        & \hyperlink{https://github.com/pan-webis-de/przybyla19/}{/pan-webis-de/przybyla19//pan-webis-de/przybyla19/}
        \\ \hline
    \end{tabular}
    \text{\footnotesize \textbf{Fonte:} Classificações obtidas de , e repositórios encontrados pelo autor.}
    \label{tab:soluções-botgender}
\end{table}